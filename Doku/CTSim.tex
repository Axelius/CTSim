\documentclass{acm_proc_article-sp}
\usepackage{times}
\usepackage{url}
\usepackage{graphics}
\usepackage{color}
\usepackage[pdftex]{hyperref}
\usepackage[ngerman]{babel}
\usepackage[babel]{csquotes}
\usepackage[utf8]{inputenc}
\hypersetup{%
pdftitle={Simulation of CT metal artefacts in C}, pdfauthor={Alexander Winkler}, pdfkeywords={your
keywords}, bookmarksnumbered, pdfstartview={FitH}, colorlinks,
citecolor=black, filecolor=black, linkcolor=black, urlcolor=black,
breaklinks=true, }
\newcommand{\comment}[1]{}
\definecolor{Orange}{rgb}{1,0.5,0}
\newcommand{\todo}[1]{\textsf{\textbf{\textcolor{Orange}{[[#1]]}}}}

\begin{document}
\setlength{\parindent}{10pt}
\setlength{\parskip}{35pt}



\title{Simulation of CT metal artefacts in C}
\numberofauthors{1}
\author{
  \alignauthor Alexander Winkler\\
    \affaddr{Chair for Computer Aided Medical Procedures \& Augmented Reality 2014}\\
    \affaddr{Technische Universit\"at M\"unchen}\\
    \email{alexander.winkler@mytum.de} }

\maketitle

\begin{abstract}
\end{abstract}

\keywords{CT, metal artefacts, simulation, segmentation, forward projection, X-Ray tube}

\section{Introduction}

\section{Causes of metal artefacts in CT}
\subsection{Fundamentals of X-Ray physics}
\subsubsection{Beam hardening}
non-linear relation between the attenuation values, \(\mu\), and the measured values of the projection
due to the fact that different bands of the frequency spectrum are differently attenuated
soft X-ray beams, are more strongly absorbed than the high-energy, hard X-ray beams. This is the reason why this effect is named hardening of the X-ray spectrum and the corresponding image error is named beam-hardening artefact.
\subsection{Fundamentals of CT reconstruction}

\section{Simulation of CT}
\subsection{Forward Projection}
\subsubsection{Overview over the forward projection}
\subsubsection{Implementation of line integrals}
\subsubsection{Implementation of beam hardening}
\subsection{Back Projection}
\subsection{Parts of the simulator}
\subsubsection{Segmented CT slice}
\subsubsection{Simulation of X-Ray tube}
\subsubsection{Look up tables for attenuation values}
%http://physics.nist.gov/PhysRefData/XrayMassCoef/ElemTab/z26.html
\subsection{Miscellaneous parts of the simulator}
\subsubsection{Logger}
\subsubsection{Image reading and writing}
\subsection{Unresolved bugs}

\section{Results}

\section{Conclusion}

\end{document}

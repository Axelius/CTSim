\chapter{Introduction}

\chapter{Causes of metal artefacts in CT}
\section{Examples of metal artefacts in CT}
\par To know why metal artefacts in CT should be simulated, we should first know what these metal artefacts are and how they look like.
\par The human body usually does not contain any metal objects. However many medical implants are composed of metal as well as many surgical tools. Thus metal can indeed be found quite often in patients who are scanned in CT scanners: Artificial hip bones, dental fillings, pacemakers, intramedullary rods, screws or in the case of inerventional CT surgical tools like needles, trocars etc. are found in many patients. If these patients are scanned in CT artefacts in the tomographic reconstruction are usually encountered, how severe these artefacts are depends mainly on their size and the material they are composed of.
\par These artefacts usually consist of star like streaks 
\section{Fundamentals of X-Ray physics}
\subsection{Beam hardening}
non-linear relation between the attenuation values, \(\mu\), and the measured values of the projection
due to the fact that different bands of the frequency spectrum are differently attenuated
soft X-ray beams, are more strongly absorbed than the high-energy, hard X-ray beams. This is the reason why this effect is named hardening of the X-ray spectrum and the corresponding image error is named beam-hardening artefact.\ref{buzug}
\section{Fundamentals of CT reconstruction}

\chapter{Simulation of CT}
par bla\ref{deMan}
\section{Forward Projection}
\subsection{Overview over the forward projection}
\subsection{Implementation of line integrals}
\subsection{Implementation of beam hardening}
\section{Back Projection}
blabla\ref{CUDABackprojection}
\section{Parts of the simulator}
\subsection{Segmented CT slice}
\subsection{Simulation of X-Ray tube}
\subsection{Look up tables for attenuation values}
blabla\ref{AttenuationTable}
\section{Miscellaneous parts of the simulator}
\subsection{Logger}
\subsection{Image reading and writing}
\section{Unresolved bugs}

\chapter{Results}

\chapter{Conclusion}
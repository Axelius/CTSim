\chapter{Introduction}

\chapter{Causes of metal artefacts in CT}
\section{Fundamentals of X-Ray physics}
\subsection{Beam hardening}
non-linear relation between the attenuation values, \(\mu\), and the measured values of the projection
due to the fact that different bands of the frequency spectrum are differently attenuated
soft X-ray beams, are more strongly absorbed than the high-energy, hard X-ray beams. This is the reason why this effect is named hardening of the X-ray spectrum and the corresponding image error is named beam-hardening artefact.\ref{buzug}
\section{Fundamentals of CT reconstruction}

\chapter{Simulation of CT}
par bla\ref{deMan}
\section{Forward Projection}
\subsection{Overview over the forward projection}
\subsection{Implementation of line integrals}
\subsection{Implementation of beam hardening}
\section{Back Projection}
blabla\ref{CUDABackprojection}
\section{Parts of the simulator}
\subsection{Segmented CT slice}
\subsection{Simulation of X-Ray tube}
\subsection{Look up tables for attenuation values}
blabla\ref{AttenuationTable}
\section{Miscellaneous parts of the simulator}
\subsection{Logger}
\subsection{Image reading and writing}
\section{Unresolved bugs}

\chapter{Results}

\chapter{Conclusion}